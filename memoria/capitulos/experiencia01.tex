% !TeX spellcheck = es_ES
\subsection{Mi paso por GKN}

Mi primer contacto se produjo a principios de Agosto de 2019, por parte de mi tutor invitándome a que pasara un día por la empresa para conocerla. Cuando acudí a visitar la fábrica me presentaron brevemente a los miembros de la oficina y me enseñaron la planta. Me sorprendió mucho el tamaño de la empresa y el alto nivel de automatización que presenta.

En Septiembre, me integraron en el departamento de IT, proyectos especiales e I+D. Durante mi estancia en el departamento, me tuvieron desarrollando nuevos proyectos, implementando nuevas características y arreglando fallos en los proyectos ya existentes. Además de enseñarnos la forma en la que daban soporte a la fábrica. Dado que en el día a día el soporte técnico a la fábrica, arreglo de bugs y desarrollo de nuevas características urgentes en los programas que ya funcionan, se come la mayoría del tiempo de los miembros del departamento, tienden a buscar el apoyo en alumnos en prácticas o subcontratando el servicio a Elara.

Los primeros días estuvieron más destinados a la colocación del puesto, explicación del funcionamiento más al detalle de la forma de trabajar de la empresa y departamento. Una vez ya estaba instalado y con una noción básica de como funcionaba el negocio me propusieron un proyecto en el que debía medir el OEE\footnote{Overall Equipment Effectivness, el cual es un ratio para medir el tiempo de trabajo efectivo en un puesto de trabajo} empleando mecanismos de visión artificial centrandose estas mediciones en la justificación de los tiempos de parada. Este proyecto lo veremos desarrollado en el apartado \ref{OEEflexcell}. Por desgracia, para este proyecto eran necesarias unas cámaras especiales, más resistentes y compactas que pidieron a una empresa Alemana, las cuales no llegaron hasta mediados de noviembre, por lo que me reubicaron para el desarrollo de otros proyectos.

Con el paso del tiempo, fueron asignándome colaboraciones en otros proyectos, sobre todo en desarrollo de nuevas características en el ERP (\ref{ERP}) y el terminal de control de accesos (\ref{controlaccesos}), a parte de otras labores relacionadas con el departamento. Estas mejoras principalmente eran lacras, que encontraban los empleados en los diferentes proyectos, pidiendo mejoras en el software.

Las últimas semanas me asignaron cumplimentar un proyecto relacionado con la medición del OEE, la celula de cascos \ref{OEEcascos}. Este puesto estaba compuesto por un robot que clasificaba automáticamente los cascos y unos terminales en los que los operarios realizaban las entradas de componentes. Estaba orientado a medir el tiempo en el que los operarios estaban declarando entradas, puesto que el sistema no realizaba ninguna medición.

La última semana estuvo más orientada a finalizar mis tareas pendientes y dejar una buena documentación. Concluir que estas semanas era cuando más agusto estaba, comenzaba a dominar las herramientas de la empresa y más integrado me sentía en el equipo. \cite{miexperiencia}

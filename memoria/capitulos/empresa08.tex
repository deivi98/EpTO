% !TeX spellcheck = es_ES
\subsection{Seguimiento del trabajo}

Los informes de seguimiento de trabajo en el departamento, se complementan mediante el sistema de tickets, cuyo funcionamiento hemos comentado en la sección toma de decisiones en el departamento de IT \ref{tomadecisionesit}.

Este sistema de tickets es la forma de justificar las horas de trabajo al grupo. El acceso a este sistema no se permite a todos los miembros del departamento, ya que deben ser autorizados por personal de otras plantas, por lo que solo una parte del departamento tiene acceso al sistema. El personal que puede acceder son los que se encargan de revisarlo, planificar cuando se van a solucionar y delegarlo al resto de miembros del equipo.

En el caso de los alumnos en prácticas, el seguimiento del trabajo lo realiza directamente por el tutor de la empresa u otros miembros del departamento, quienes asignan tarea y realizan su supervisión.

En mi opinión no se desarrolla un mecanismo de seguimiento de trabajo organizado, aunque de esta forma se reduce el estrés y se aumenta la libertad de los miembros del departamento. Además dada la naturaleza del soporte técnico, los tiempos de resolución no son fácilmente calculables. También añadir que se evita la perdida de tiempo en rellenar partes de trabajo, centrándose en la producción.

Por último añadir que las metodologías de gestión de proyecto modernas, como Scrum o Kanban, se adaptan perfectamente a la gestión de empresas de software. Sin embargo Scrum no lo veo muy adecuado para la parte de soporte técnico, cosa que Kanban si que es capaz de gestionar correctamente. Hemos hablado sobre estas técnicas en la sección de proyectos \ref{proyectos}. \cite{miexperiencia}
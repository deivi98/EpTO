% !TeX spellcheck = es_ES
\subsection{Toma de decisiones en el departamento de IT}

\subsubsection{Funciones departamento}

Para hablar de la toma de decisiones, considero oportuno explicar primero cómo funciona el departamento de IT. En el trabajan de forma continua 4 personas: director: se dedica a la planificación de nuevos proyectos, comunicación con servicios externos y realizar de supply chain para la fábrica; Un administrador de sistemas y otros 2 integrantes enfocados en el desarrollo de software y soporte técnico a la fábrica. Al ser tan pequeño, todos colaboran en las labores de los demás si es necesario. Añadir que durante mi estancia eramos 6 personas, ya que realizamos prácticas 2 alumnos de la UPNA entre Septiembre y Noviembre.

A nivel de grupo el departamento de IT en las plantas de producción, teóricamente debe dedicarse a dar soporte técnico a los usuarios(hacer que funcionen las impresoras, sustitución e instalación de equipos, gestión del ERP\footnote{Enterprise Resource Planing, Sistema de gestión empresarial en castellano} genérico...), las labores de configuración de servidores, creación de módulos de SAP, gestión de diferentes servicios ligados a cada usuario y otras funcionalidades, se delegan a gente especializada en el grupo que están en otras plantas. También se limita el número de miembros del departamento en función del tamaño de la planta.

A nivel de planta, hace 20 años y dada la complejidad creciente del negocio, se tomó la decisión de dejar de usar SAP como ERP, para comenzar a desarrollar la planta su propio ERP, manteniendo SAP únicamente para el departamento de finanzas(ya que este es el más genérico en cualquier empresa). Por tanto, hoy en día el departamento de IT se dedica a dar soporte técnico a la planta y al desarrollo de nuevas funcionalidades en el ERP. También añadir que ha habido intentos por parte del grupo, de implantar SAP como ERP común en todas las plantas, poniendo en peligro gran parte del trabajo realizado en esta aplicación, ya que el cambio de un ERP puede suponer más de 2 años de trabajo.

Por último, destacar que el departamento de IT está unido a proyectos especiales e I+D, por lo que el personal del departamento esta presente en el desarrollo de nuevos proyectos especiales y en la implantación de tecnología en la planta. \cite{miexperiencia,planacogida}

\subsubsection{Toma de decisiones}
\label{tomadecisionesit}

Las decisiones en el departamento de IT se realizan mediante el sistema de tickets, en el que los usuarios reportan un problema o proponen una mejora en los servicios. Esta es evaluada automáticamente por el sistema, asignándole una prioridad en función del número de usuarios afectados y su gravedad. Por ejemplo, un ticket que afecta a un usuario y no le impide trabajar tendrá la prioridad más baja, en cambio, uno que afecta a toda la planta y no les permite trabajar tendrá la máxima prioridad.

En el día a día, se da prioridad a los bugs y fallos en sistemas sobre el desarrollo de nuevas características. Al ser una planta grande y automatizada, estos son frecuentes, por tanto, el desarrollo de nuevas características tiende a  retrasarse. Para nuevos proyectos de desarrollo a largo plazo, se cuenta con el apoyo de alumnos en prácticas, los cuales desarrollan la base, hasta que se acercan la fase de implantación, momento en el que se colabora más estrechamente con el resto del departamento, para implantarlo correctamente y pueda ser continuado una vez finalizadas las prácticas. \cite{miexperiencia}



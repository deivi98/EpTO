% !TeX spellcheck = es_ES
\paragraph{El puesto}
La máquina de control de accesos es una terminal situada en la entrada de la fábrica donde nos debemos registrar para que se autorice la entrada de personal ajeno. Este sistema distingue entre visitas y personal de empresas subcontratadas, el cual ha entregado previamente la documentación en GKN.

Esta aplicación esta escrita en Visual Basic, con formularios WTF y gestión de datos en Oracle.

\paragraph{Problema}
En este caso, me tocó enfrentarme a dos problemas. Por un lado, el departamento de seguridad realizó una queja, ya que el personal de empresas subcontratadas accedía a la fábrica como una visita, con lo cual no se comprobaba que los papeles de esta empresa estuviesen al día. Por otro lado, los usuarios se quejaron de la interfaz de registro, ya que era demasiado engorrosa y se pidió su simplificación.

\paragraph{Mi participación}
Este proyecto, fue el TFG de un alumno anterior, por lo que mi parte fue su modificación para paliar las quejas del departamento de seguridad y usuarios.


\paragraph{Desarrollo del problema}
Para no permitir que empleados de subcontratas se registrasen como visitas tuve que cambiar el registro de estas, añadiendo la comprobación del DNI de cada persona no estuviese registrado en una empresa subcontratada. También modifique el registro de empresas para que se comprobara que toda la documentación estuviese al día. Ya que me tocó realizar un cambió estructural bastante grande, aproveché para mejorar el mantenimiento del código ante futuras modificaciones. 

Al tiempo, el departamento de seguridad volvió a pedir la simplificación de la interfaz de registro por las quejas realizadas por lo usuarios. Además aprovecharon para pedir otras mejoras en esta: eliminación de características que ya no se usaban, actualización de los datos da la aplicación y solución de algún bug.

Para mejorar esto, tuve que eliminar los formularios que ya no se usaban y hacer los existentes más agradables para el usuario y que no le resultara tan pesada la introducción de datos. Además, tuve que actualizar los datos de empleados en la BD para que se mostrasen correctamente.


\paragraph{Conclusión}
La realización de este proyecto fue muy similar a los cambios pedidos en el ERP, con la excepción de que me resulto muy útil para ver nuevos entornos gráficos. En concreto, los formularios WTF, los cuales se están usando mucho actualmente por su flexibilidad.


% !TeX spellcheck = es_ES
\paragraph{El puesto}
La célula flexible(flexcell) es una serie de máquinas orientada a la fabricación de manguetas nuevas en series cortas. Estas están distribuidas de tal forma que optimiza los tiempos de cambios de utillaje, con un diseño modular que permite el uso de solo determinadas zonas a demanda.

\paragraph{Problema}
En Carcastillo a diferencia de otras plantas se ha especializado en la realización de series cortas, por lo que el cambio de utillaje y material de fabricación se realiza con una frecuencia de unas 30 veces mayor que en el resto de plantas. El sistema para medir el OEE que dispone la empresa no es capaz de medir estos tiempos automáticamente, teniendo que ser el operario el que justifica los tiempos no productivos, cosa que es posible que se olvide, o no, del tiempo a su realización.

\paragraph{Solución propuesta}
Se pretende realizar la medición de los eventos del OEE automáticamente sin la dependencia de usuarios. Para ello se implantarán una serie de cámaras en posiciones estratégicas, y sea una inteligencia artificial la encargada de clasificar estos eventos. Los esfuerzos se centrarán principalmente en cuantificar los tiempos no productivos, para poder mejorar el tiempo en estos: cambios de utillaje, averías...

Los resultados obtenidos se usaran para mejorar la toma de decisiones para mejorar el rendimiento de la célula, introduciendo más operarios, mejorando su formación en ciertos aspectos, cambiando maquinaria...

Para ello la visión artificial va a jugar un papel fundamental, ya que, mediante lo captado por las cámaras, nos dará los eventos de la célula en todo momento.

\paragraph{Mi participación}
En este proyecto, yo he sido el primero en desarrollarlo y he hecho el planteamiento inicial sobre el clasificado de imágenes. 

Por problemas de logística, las cámaras necesarias para realizar estas mediciones no llegaron a tiempo, por lo que realicé las pruebas sobre otra célula, el almacén automático de servicio. Este problema contaba únicamente con una cámara enfocando la célula y se trataban de cajas con utillaje en un flujo continuo, por lo que, el problema cambia respecto al original, puesto que no vemos una máquina trabajando en todo momento como puede ser un torno.

Dadas las circunstancias, me dediqué a estudiar diferentes ``papers'' sobre la medición del OEE y detección de objetos en la industria, planteando los diferentes caminos mediante los que se puede realizar este proyecto y realizando especulaciones sobre los resultados esperados por cada solución.

\paragraph{Planteamiento de soluciones para la medición}

Las ideas planteadas para solucionar este problema fueron las siguientes:
\begin{itemize}
	\item Basarme en unos pixeles específicos para el clasificado.
	\item Detección de objetos y seguimientos de los mismos.
	\item Pasar a una red neuronal una secuencia de imágenes de las diferentes cámaras.
	\item Restar imágenes secuenciales para detectar las zonas en las que ha habido movimiento.
\end{itemize}

El \textbf{clasificado basado en pixeles clave} consiste en seleccionar los pixeles donde se desarrolla una acción continua y midiendo cuándo ha habido un cambio. Con éste, podríamos realizar el conteo de piezas que pasan por una cinta transportadora. Dicho sistema es muy sensible a los cambios de iluminación producidos por la domótica y poco extensible, por lo que descarté la solución en una fase temprana.

La \textbf{detección de objetos y su posterior seguimiento} se trata de una solución con muy buenas expectativas. La idea base era detectar todos los objetos importantes captados por una cámara, estudiar su movimiento en imágenes consecutivas y con estos datos realizar su clasificado. Para implantarlo sería necesario el desarrollo de una red neuronal de 3 capas. La primera capa se encargaría de la detección de objetos(operarios, llaves inglesas, piezas de utillaje, brazo de un robot...), la segunda capa mediante los objetos detectados en imágenes consecutivas se dedicaría a trazar su ruta, y una última capa, que en función de las rutas de los objetos detectados, realizaría una clasificación del evento que está ocurriendo en esa secuencia de imágenes.

\begin{figure}[h]
	\caption{Esquema de detección de estados}
	\centering
	\label{esquemaflexcell}
	\includegraphics[width=0.9\textwidth]{esquemaProcesamiento.png}
\end{figure}

Dicha solución la dejamos de desarrollar debido al tiempo que costaría su implantación, ya que eran necesarios unos 3 meses de trabajo para la toma de imágenes y su clasificación, una estación de trabajo especializada en el procesamiento gráfico para ser capaces de realizar el entrenamiento y clasificar las imágenes en un tiempo aceptable, además del tiempo necesario para el desarrollo de los algoritmos. 

Esta opción fue explorada hasta lograr detectar objetos con la red neuronal de yolo\footnote{You only look once, algoritmo para la detección rápida de objetos} y no se siguió implementando ya que el problema puede cambiar mucho de una célula a otra. También otro de los impedimentos de este proyecto fue la baja flexibilidad, ya que el tiempo de captura de datos y entrenamiento se debía repetir para implantarlo en otra célula.

Otra de las opciones era pasar una \textbf{secuencia de imágenes directamente a una red neuronal}, esta red neuronal podría ser sustituida perfectamente por un clasificador bayesano. Para la implantación de este método sería necesaria la creación de un dataset de un tamaño considerable, debido a la gran dimensionalidad de los datos, ya que cada imagen está compuesta por 800x600x3 bytes. Además necesitaríamos tomar las imágenes de 3 cámaras y coger las últimas 3 imágenes  que se han sacado. Presentándonos ante una entrada de unas 13 millones de características. Para que esta solución fuese viable, sería necesario reducir el número de características, para ello podríamos convertir la imagen a escala de grises, reducir el número de pixeles y usar las redes neuronales de pre procesamiento de imágenes ya implementadas.

Por último, la opción más sencilla, aunque con resultados no tan prometedores, es realizar la \textbf{resta de imágenes secuenciales}, obteniendo un mapa de calor de las zonas en las que se está produciendo movimiento. Para ello, necesitaríamos hacer la resta de los pixeles de dos imágenes secuenciales, aplicar un filtro sal y pimienta, y realizar el suavizado de la mediana. La nueva imagen obtenida deberíamos aplicarla a un algoritmo de clasificado. Este método es interesante como preprocesamiento para el método explicado con anterioridad.

\paragraph{Análisis del proyecto}
A la hora de hacer este proyecto realizamos un análisis DAFO. En él, estudiamos las fortalezas y oportunidades que nos daba la realización de este proyecto, además de sus consecuentes amenazas y debilidades.

\begin{figure}[h]
	\caption{Análisis DAFO flexcell}
	\centering
	\label{dafoflexcell}
	\includegraphics[width=0.95\textwidth]{dafo.jpg}
\end{figure}

\paragraph{Conclusión}
Estamos ante un proyecto interesante e innovador el cual explota características que los sistemas tradicionales no son capaces de medir o requieren de un numero absurdo de sensores para poder llegar a diagnosticar las causas de una parada en la máquina.

El problema en si es muy complejo y requiere muchas horas de trabajo, sobre todo en la toma, etiquetado y preprocesamiento de los datos, el cual puede suponer un 70\% del tiempo dedicado al proyecto para tener un conjunto de datos lo suficientemente grande para que nos pueda decir con un buen nivel de fiabilidad los eventos.

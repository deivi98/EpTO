% !TeX spellcheck = es_ES
\paragraph{El programa}
GKN Carcastillo cuenta con su propio sistema ERP desarrollado a medida para ellos. Este se comenzó hace unos 15 años y en sus comienzos se realizó mediante la empresa subcontratada Elara. Con el paso del tiempo y evolución del proyecto, se vio que la dependencia de empresas externas para el desarrollo del software, no era lo suficientemente ágil para los requisitos del negocio, por lo que poco a poco se fue trasladando el desarrollo del software al departamento de IT.

En la actualidad la gestión de todos los aspectos de la empresa, salvo finanzas se realiza mediante este ERP. El sistema está desarrollado en el lenguaje Basic, en el entorno de .NET, utiliza las bases de datos de Oracle para la gestión de datos y Crystal Reports para la generación de informes. 

\paragraph{Mi participación}
En este proyecto he contribuido en el desarrollo de nuevas características y solución de bugs. Estas tareas venían determinadas por tickets que habían realizado los diferentes departamentos de la empresa y me eran asignados por mi tutor o el director del departamento.

En este proyecto es donde he realizado la mayor parte de mi esfuerzo y me ha tocado trabajar con todos los aspectos del proyecto: desarrollo de código en VB, realización de formularios, consultas a la BD e informes Crystal Reports. También me ha tocado trabajar con varios departamentos: Seguridad, Compras, Planificación de la producción e Ingeniería de fabricación avanzada.

En cuanto a las tareas realizadas, estas eran de índoles muy diversas y de corta duración, por lo que, detallar cada una de ellas sería demasiado extenso y poco interesante para el contenido de la memoria, por lo tanto el detalle de cada uno de estos cambios no está plasmado en la memoria.

\paragraph{Conclusión}
Trabajando en un programa que se comenzó a escribir hace 15 años, en el cual han intervenido muchas personas diferentes, he visto la gran importancia de usar un código limpio y extensible, ya que cambios triviales en un programa que no cumple estas características, pueden ver multiplicado su tiempo de realización por un factor de 20, ya que puede ser necesario reescribir todo el código según como se haya implementado. También he visto la importancia de herramientas de control de versiones en las cuales es muy fácil seguir los cambios del proyecto y descartar los que causen problemas.

También, me ha tocado ver cómo los requisitos iniciales propuestos para un proyecto no son fijos y estos casi siempre, acaban cambiando cuando le muestras la beta al usuario final, puesto que, siempre hay detalles que no se han mencionado o aparecen al enseñar la beta. Además, no siempre es posible darle al usuario lo que pide, ya que se debe adaptar a la infraestructura de un programa ya creado, por lo que te ves obligado a ofrecerle alternativas que realicen el mismo fin.

Por último, he visto las ventajas de un ERP a medida, cuyo coste de implantación inicial es muchísimo mayor que un ERP genérico, pero la flexibilidad que aporta al negocio es un aspecto clave para que una empresa pueda adaptarse a tiempo al mercado.

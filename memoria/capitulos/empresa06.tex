% !TeX spellcheck = es_ES
\subsection{Técnicas de motivación y retención de personal}

Estamos ante una empresa que está lejos de los principales núcleos de población, por lo que corre un alto riesgo de fuga de personal a centros de trabajo más cercanos a sus hogares. Es por ello que la empresa ha comprendido muy bien lo valioso que es el capital humano y lo aprecian mucho. Estas son algunas medidas que toman para evitar la fuga de sus empleados y mantenerlos motivados:

\begin{itemize}
	\item \textbf{Buen ambiente empresarial, } el ambiente en la oficina es muy bueno y casi todo el mundo está abierto a ayudarte si lo pides. No les importa ``perder'' el tiempo en ti, sobre todo dentro del propio departamento.
	\item \textbf{Inversión en formación, } la empresa invierte tiempo y dinero en la formación de sus trabajadores si esta repercute aportando más valor a la empresa, por tanto es una buena oportunidad para especializarte.
	\item \textbf{Horarios, } la empresa cierra en épocas donde suele ser difícil escoger vacaciones, como son navidad, semana santa y fiestas de Carcastillo, además pone días de libre elección para el trabajador y si tu puesto te lo permite puedes tener más libertad horaria.
	\item \textbf{Metas alcanzables y primas, } la planta establece metas a cumplir, las cuales no son imposibles de alcanzar logrando motivar a sus empleados con su cumplimento. Además establece primas de producción a los operarios de fábrica.
	\item \textbf{Salarios, } la empresa cuenta con convenio colectivo propio, que mejora las condiciones del convenio del colectivo del metal(considerado uno de los mejores actualmente), por lo que es un buen incentivo a seguir.
	\item \textbf{Posibilidad de escalado, } Al ser una multinacional, las posiciones a las que ascender son muy grandes.
	\item \textbf{Continua mejora de la planta, } ver como mejora tu fábrica y sentirte parte del cambio puede ser algo muy reconfortante, pues te hace sentir parte de la mejora.
	\item \textbf{Garantía de longevidad, } al ser una empresa que no ha parado de crecer desde su fundación, es muy probable la obtención de un trabajo estable, ya que las probabilidades de surgimiento de nuevos puestos es mayor.
	\item \textbf{Sin jefes, solo lideres, } la empresa no está sometiendo una continua presión laboral a sus trabajadores como haría un jefe, cuenta con lideres que están trabajando codo con codo contigo y te permiten sacar lo mejor de ti.
\end{itemize}

\paragraph{Mi opinión} En mi opinión es un muy buen centro de trabajo, el cual cuenta con las ventajas de ser una gran empresa al ser una multinacional, manteniendo la esencia de una mediana en la que es muy fácil conocer a todos sus empleados y donde se fomenta el buen ambiente laboral.

También se ve una alta presencia de personal de la zona de Carcastillo y alrededores, al ser estos pueblos pequeños muchas personas se conocían antes de entrar en la empresa, fomentando el compañerismo.

Otra cosa que me ha gustado mucho ha sido el alto grado de automatización, junto a la inversión en mejoras e I+D. Ya que se puede apreciar un cambio radical en el aspecto de la empresa en apenas 5 años, haciendo que los empleados que estaban ahí desde el principio sientan el proyecto de empresa como su proyecto personal.

Por último, señalar la calidad de horarios y salario, lo cual es uno de los factores principales que puede llevar a cabo una empresa para la retención de personal. \cite{miexperiencia}

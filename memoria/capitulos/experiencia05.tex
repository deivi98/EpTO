% !TeX spellcheck = es_ES
\subsection{Recursos}

\subsubsection{A nivel de planta}

En el caso de GKN, la política de grupo no permite tener equipos de más de 5 años en activo, por lo que todos los equipos de la planta están en estado óptimo y con un buen rendimiento. Por ello no debería haber quejas de ningún usuario por los equipos informáticos proporcionados.

En caso de que algún usuario solicite alguna necesidad especial derivada de sus funciones, como puede ser un equipo con una buena tarjeta gráfica para el desarrollo de modelos en 3D, mejores procesadores para compilar más rápido para desarrollo de código o escaners para leer códigos de barras más rápido, se les da dicha necesidad. Además, si esta es una mejora de productividad global a nivel de departamento, también se implanta en la medida de lo posible.

Para la zona de producción, la empresa cuenta con un gran nivel de automatización, apreciándose en el gran número de robots y equipos informáticos que hay a lo largo de la planta.

Por último, decir que el grupo tiene un contrato global con Dell, por lo que todos los equipos son de la misma marca y es más rápida y sencilla su conexión entre sí.

\subsubsection{Mis recursos}

En mi caso, para el desarrollo del software, se me proporcionó un portátil con una pantalla auxiliar y los periféricos correspondientes. Material más que suficiente para el desarrollo del mismo.

Para el proyecto de la flexcell, requerí de un equipo con mejor procesamiento gráfico, por lo que me dejaron un equipo con una tarjeta Quadro P4000 y Sistema operativo Linux, para realizar el entrenamiento de la red neuronal en un tiempo menor. También, para este proyecto fue necesaria la instalación de cámaras de videovigilancia en la célula. Estas se pidieron antes de que entrara en la empresa y no fue hasta finales de noviembre cuando las recibió, por lo que nos vimos obligados a dejar en Stand-by este proyecto. Debido a ello, me reubicaron para colaborar con otros proyectos del departamento.

Para concluir, me hubiese gustado mucho desarrollar más el proyecto de la flexcell, ya que sin las cámaras lo único que pude hacer fueron especulaciones y un estudio de métodos para hacerlos, sin poder llevarlos a la práctica.








  
% !TeX spellcheck = es_ES
\subsection{Descripción e Historia}
\subsubsection{El grupo GKN}
\paragraph{Descripción}

\begin{wrapfigure}{r}{0.33\textwidth} %this figure will be at the right
	\caption{Que hace}
	\centering
	\includegraphics[width=0.33\textwidth]{queHace.jpg}
\end{wrapfigure}

GKN (Guest, Keen and Nettlefolds) es una empresa británica dedicada al sector del metal, en concreto a la fabricación de componentes automovilísticos y aeroespaciales.

Nos encontramos ante una multinacional la cual cuenta con más de 55000 empleados y presencia en más de 30 países, con un volumen de ventas de 10.07B€ en el año 2018.

Las principal división de la empresa es el sector automovilístico, donde el grupo emplea a más de 29000 personas repartidas en 54 fábricas a lo largo de 21 países y 5 centros globales de tecnología, división que facturo 5.73B€ en el 2018. La otra división principal es el sector aeroespacial el cual se encuentra en 50 fabricas distribuidas en 15 países diferentes donde trabajan más de 18000 personas la cual vendió en 2018 4,12 billones de euros. El resto de divisiones se encargan de la fabricación de ruedas, estructuras, piezas metalúrgicas e incluso cuenta con una división dedicada a la Formula E donde el grupo patrocina al equipo "Panasonic Jaguar Racing". \cite{wikipedia-gkn,web-gkn,forbes-gkn}

\paragraph{Historia}


Podemos encontrar las primeras referencias de la empresa en libros en el año 1754, con las referencias encontradas a Dowlais Iron Company en la villa de Dowlais, Merthyr Tydfil, Wales, Reino unido. Aunque no será hasta 1902 cuando veamos aparecer por primera vez Guest, Keen and Nettlefolds con la fusión de Dowlais Ironworks y Patent Nut \& Bolt Co. Ltd en 1900 y la adquisición de Nettlefolds Ltd en 1902. A partir de ese momento GKN no paro de crecer fusionandose y adquiriendo más empresas como podemos ver en la Figura \ref{crono}.

\begin{wrapfigure}{L}{0.23\textwidth} %this figure will be at the right
	\caption{comienzos GKN}
	\centering
	\includegraphics[width=0.23\textwidth]{oldimage.PNG}
\end{wrapfigure}

La empresa surgida durante la revolución industrial vio en el ferrocarril una buena oportunidad de crecimiento y gracias a Sir Jhon Guest, la empresa fue la principal proveedora de metal del Ferrocarril del Gran Oeste en Gran Bretaña(1838). También importaron sus piezas metálicas globalmente llegando a vender 50000 toneladas de vías a Rusia lo que permitió posicionarse a Dowlais Ironworks como la mayor compañía metalúrgica en el mundo en el año 1845.

En 1852 tras la muerte de Sir Jhon Guest fue su mujer Lady Charlotte Guest quien asumió la dirección de GKN, la cual no estaba pasando por buenos momentos: el negocio se había vuelto demasiado complejo, las condiciones de trabajo no eran las óptimas y había problemas con el arrendamiento de las fábricas, fue entonces cuando Charrlotte a la edad de 40 y con 10 hijos a su cargo asumió el negocio durante estos críticos años. Durante su mandato convirtió Dowlaris Ironworks en la compañía de fabricación más grande del mundo y estableció planes para su futura expansión, incluso aprendió galés para poder hablar con sus trabajadores.


%\begin{wrapfigure}{L}{0.25\textwidth} %this figure will be at the right
%	\caption{Producción metal}
%	\centering
%	\includegraphics[width=0.25\textwidth]{steel-production.jpg}
%\end{wrapfigure}

En 1856 Dowlais fue la primera empresa en adquirir una licencia para trabajar con el metal, fue el fin de una década de experimentación en GKN y la fijación de un rumbo fijo. La producción de metal fue creciendo hasta producir 26000 toneladas en 1871 y más de 118000 en 1884, adentrándose en el siglo 20 como una de las principales fundiciones del mundo.

Con la llegada de la primera guerra mundial, GKN cambió para satisfacer la demanda bélica del país, por lo que se dedicó a la producción de acero para las diferentes aplicaciones militares. Las principales contribuciones de la empresa para la guerra fueron balas(en su mayoría), forjas, sujetadores para tanques y aviones, así como millones de cascos de acero. Tras ello comenzaron a fabricar componentes de motor.

GKN emergió de la guerra como el mayor productor de acero de Gran Bretaña, aunque pronto se vio envuelto en una batalla contra el gobierno laborista, que quería nacionalizar todas las fábricas. En 1951 dicho gobierno nacionalizó la industria del acero pagando a GKN £18.5M por sus activos. Cuatro años después estas políticas fracasaron y GKN recuperó sus activos pagando por ellos menos de £12M.

A mediados del siglo 20 nacen los aires diversificadores de GKN lo que marcaría el rumbo hasta la actualidad, ampliando sus dominios con la compra y creación de empresas. En 1960 se funda GKN automotive con la adquisición de Birfield automotive components group. En 1982 con la compra de Westland plc, comienza GKN aerospace y se cambia el nombre de la empresa a GKN plc. A partir de este momento la historia de GKN con 4 divisiones diferenciadas comienza a tomar diferentes rumbos cuyo estudio no procede en esta memoria. \cite{web-gkn,gkn-history} 

\begin{figure}[h]
	\caption{Árbol genealógico de GKN, 1914 a 1945}
	\centering
	\label{crono}
	\includegraphics[width=0.7\textwidth]{gkn1914-1945.PNG}
\end{figure}

\begin{table}
	\caption{Cronologia del grupo GKN}
	\scalebox{1}{
	\begin{tabular}{r |@{\foo} l}
		
		1754 & Primeras referencias a GKN con Dowlais Iron Company\\
		1838 & Un camino rápido hacia el éxito, llega el ferrocarril \\
		1852 & Lady Charlotte Guest una mujer pionera, líder de la empresa \\
		1856 & Del hierro al acero \\
		1902 & Creación de Guest, Keen and Nettlefolds\\
		1925 & Balas y motores de coches, GKN suministrador durante la guerra mundial \\
		1950 & Nacionalización y diversificación\\
		1960 & El camino a la automoción, se crea GKN automotive\\
		1982 & Tomando nuevos rumbos: GKN plc\\
		2018 & Melrose compra GKN
		
	\end{tabular}
}
\end{table}

\break

\subsubsection{GKN Driveline Carcastillo}
\paragraph{Descripción}

GKN Driveline Carcastillo es una planta perteneciente al grupo GKN, concretamente a la división Driveline. Está ubicada entre Pamplona y Tudela en el municipio de Carcastillo. Se trata de una empresa en continuo crecimiento con una constante inversión en mejorar, además cabe destacar el alto grado de automatización con el que cuentan sus instalaciones.

Esta planta se dedica principalmente a la refabricación de transmisiones, aunque en los últimos años se ha optado por el diseño y fabricación de componentes, especializándose en las series cortas que otras plantas del grupo y competencia no realiza debido a su menor rentabilidad. Este servicio esta muy demandado por empresas automovilísticas de lujo como Ferrari y Bentley, para las que se fabrican recambios bajo el sello OEM\footnote{OEM (Original Equipment Manufacturer): Fabricante de equipo original} en Carcastillo.

Dado que mi experiencia ha sido unicamente con la planta de Carcastillo, el resto del informe se verá enfocado en esta planta y no en el grupo.

\paragraph{Historia}

La empresa se creó para el refabricado de transmisiones, cuando el mercado empezaba a ampliarse en otros países y la multinacional GKN quiso potenciarlo también en España. Dado que la compañía contaba ya con otras dos plantas en Zumaia y Vigo para la fabricación de transmisiones nuevas, con esta nueva planta cerrarían el ciclo, reciclando y reutilizandolas una vez finalizada su vida útil, llevándolas al mercado de postventa.

Gracias a la relación que mantenían por aquel entonces el alcalde de Carcastillo y uno de los directivos de GKN, ofreció a la multinacional todas las facilidades para que la instalación se acabase realizando en Carcastillo, consolidándose con la construcción de la planta en 1987.

En 2008 el almacén de servicio del grupo de Barcelona se traslada a Carcastillo, convirtiéndose la planta en un importante punto de distribución del grupo.

En los último años se ha invertido en la fabricación de componentes, realizando transmisiones y juntas hechas 100\% en Carcastillo (incluyendo el diseño). Se ha especializado sobre todo en la producción de series cortas cuya demanda no se satisfacía a tiempo por la competencia, creando una célula flexible, diseñada para facilitar el cambio de utillaje y material. 

La infraestructura también se ha ido ampliando con el crecimiento de la planta, la última gran obra fue la construcción de una nueva zona de oficinas, satisfaciendo la creciente demanda de espacio. En la actualidad se esta construyendo un nuevo almacén de distribución anexo a la planta, usando este como almacén de servicio y el antiguo como centro de distribución de material en producción. \cite{miexperiencia,planacogida,transmitiendoorgullo} 

\begin{table}[h]
	\caption{Cronologia de GKN Driveline Carcastillo}
	\scalebox{0.85}{
		\begin{tabular}{r |@{\foo} l}
			1987 & Se crea Gkn Ayra Servicio SA \\
			1995 & Comienza el remanufafacturing de juntas de bolas, CVJ (Constant Velocity Ball Joints)\\
			1998 & Se traslada el remanufacting desde Offenbach a Carcastillo y Ribemont \\
			2008 & El almacén de GKN Ayra Servicio de Barcelona se traslada a Carcastillo\\
			2009 & Implantación de clasificado automático de cascos \\
			2009 & La planta de Carcastillo suministra el 100\% del mercado independiente mundial\\
			2013 & Se implante el remanufacturing y montaje de Propshaft\\
			2013 & La planta alcanza los 100 empleados\\
			2015 & Primer suministro de OEM para Bentley y Ferrari\\
			2017 & Se comienza el proyecto de célula flexible para lotes cortos\\
			2018 & Melrose compra GKN\\
			2018 & La planta de Carcastillo alcanza los 150 empleados
		\end{tabular}
	}
\end{table}



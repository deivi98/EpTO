% !TeX spellcheck = es_ES
\subsection{Posicionamiento en el mercado}

\subsubsection{Ámbito}

Dado que estamos con una empresa con presencia en los 5 continentes, claramente GKN plc es de ámbito internacional. En cuanto a la planta de Carcastillo, las transmisiones fabricadas y refabricadas, están orientadas principalmente al mercado Europeo, aunque llegan a suministrarse en todo el mundo, por tanto el ámbito de aplicación de GKN Driveline Carcastillo también es internacional. \cite{planacogida}

\subsubsection{Necesidad y posicionamiento} 

\begin{wrapfigure}{R}{0.45\textwidth} %this figure will be at the right
	\caption{Ciclo de vida transmisión}
	\centering
	\includegraphics[width=0.23\textwidth]{ciclo.png}
\end{wrapfigure}

GKN Driveline Carcastillo nació con el fin de completar el ciclo de vida de las transmisiones, reutilizando aquellas que ya habían completado su ciclo de vida y convirtiéndolas en transmisiones nuevas con la etiqueta de refabricado. Cubriendo así la necesidad de reutilizar las transmisiones viejas, además de verse reducido el impacto medioambiental mediante este reciclaje.

Con la experiencia, se ha ido diversificando, cubriendo otras necesidades del grupo. En 2008 con el traslado del almacén de servicio del grupo de Barcelona a Carcastillo, se comenzó a desempeñar esta labor y en 2014 se mejoro su desempeño, con el desarrollo de un almacén automático: compuesto por un transelevador y 1592 cajas de 200 kg.

En los últimos años, la fábrica vio la oportunidad en las series cortas, las cuales se dejaban de servir a tiempo en el mercado dados sus bajos margenes de beneficios. GKN Carcastillo desarrolló una célula especializada con este fin: la flexcell \footnote{Célula flexible}, diseñada de forma modular para adaptarse al mayor número de transmisiones, además los operarios están entrenados para realizar los cambios de utillaje lo más rápido posible, puesto que se dan a diario, mientras que en otras plantas se realizan una o dos veces al mes.\cite{miexperiencia, planacogida, planestrategico, mecaluxrobot}

\subsubsection{Competencia}

El grupo GKN incentiva la competencia entre plantas, por tanto el competidor principal es la planta Ribemont, la otra en Europa que tiene el fin de reutilizar las transmisiones usadas. Con la incorporación de la célula flexible y otros puestos de fabricación, ha comenzado a competir con las plantas de producción del grupo, aunque realmente les libera de las series cortas, cuya producción solía ocasionar problemas para cumplir con los plazos.

En cuanto a competidores externos, los principales son: \emph{American Axle \& Manufacturing Inc, The NORDAM Group Inc y Linamar Corporation.} \cite{miexperiencia, zoominfo}


\break
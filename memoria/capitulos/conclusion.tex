% !TeX spellcheck = es_ES

GKN es una empresa multinacional dedicada a la fabricación de piezas de automóviles para el sector aeroespacial y soluciones terrestres. La planta de Carcastillo que se trata en esta memoria, se dedica al reacondicionamiento de transmisiones usadas. Esta planta ha ido creciendo con los años hasta llegar a los 150 empleados y comenzar a diversificarse, adquiriendo nuevas competencias, como hacer de almacén de servicio del grupo, o fabricar sus propias piezas.

Sobre GKN Driveline Carcastillo, es una empresa que tiene en gran consideración a sus trabajadores y es consciente de que este es uno de sus activos más importantes. Además, vemos en esta planta una inversión continua en el desarrollo de las instalaciones, con una mejora contante en el proceso de fabricación. Por ello, estamos ante una empresa muy competitiva.

Acerca de mi experiencia, he estado trabajando en un pequeño departamento de IT dedicado a dar soporte a la planta, haciendo que todos los equipos funcionasen correctamente, además del desarrollo de software necesario para el negocio. En mi caso, me tocó dedicarme al desarrollo de software, realizando el planteamiento de un nuevo proyecto y colaborando en la mejora de los ya existentes.

En mi estancia, me ha tocado trabajar con diferentes lenguajes de programación y herramientas. Principalmente he trabajado con SQL, python, powershell y basic. Además, me ha tocado aplicar los conocimientos teóricos vistos en la universidad fuera de un entorno simulado en el que he podido hablar directamente con el cliente que requiere el software, siendo este, otro empleado de la empresa. Lo más interesante de esto, ha sido ver como los requisitos iniciales casi siempre cambiaban cuando el cliente veía una nueva versión.

Sobre el funcionamiento del departamento de IT, al ser tan pequeño era muy fácil buscar ayuda y darla, por lo que rápidamente me sentí un miembro más del equipo. En este no se realiza una gestión de tareas, sino que es el director del departamento o el tutor, el que se encarga de distribuirlas, la cual, podría mejorarse mucho aplicando kanban, sobre todo si crece el tamaño del departamento, cosa que es probable debido a la carga de trabajo actual.

Me ha gustado mucho mi estancia en GKN, sin embargo he decidido cambiar, debido a que ya he estado en dos grandes empresas: NASERTIC y GKN, con un pequeño departamento de IT para dar soporte a las necesidades del negocio, por lo que ahora me apetece conocer el funcionamiento una empresa dedicada exclusivamente al desarrollo de soluciones informáticas.

Por último, concluir que recomendaría GKN para futuras prácticas, ya que en ella te cuidan muy bien y representa la parte de la informática de dar soporte a una empresa, lo cual refleja una gran parte de las ofertas de trabajo del sector.




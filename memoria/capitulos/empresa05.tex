% !TeX spellcheck = es_ES
\subsection{Proyectos}
\label{proyectos}

En el caso de GKN Carcastillo, cada departamento se encarga de la planificación de sus proyectos internos. Dado que no tengo experiencia en cómo se organizan otros departamentos, voy a hablar únicamente del departamento de informática.

El seguimiento de proyectos se hace mediante una pizarra, en la que están expuestos todos los proyectos abiertos. A cada proyecto se asigna un encargado y se escribe a lado su proceso. Junto a esta tabla, existen otras en las que cada empleado escribe los proyectos en los que está trabajando y su estado: En producción, stand by o parado.

\begin{table}[ht]
	\centering
	\caption{Planificador de proyectos}
	\begin{tabular}{|l|l|l|llllll}
		\cline{1-3} \cline{5-6} \cline{8-9}
		\multicolumn{3}{|l|}{PROYECTOS IT} & \multicolumn{1}{l|}{} & \multicolumn{2}{l|}{Proyectos Raimundo Pérez(RP)}                 & \multicolumn{1}{l|}{} & \multicolumn{2}{l|}{Proyectos Pablo Garcia(PG)}                   \\ \cline{1-3} \cline{5-6} \cline{8-9} 
		Proyecto 1     & RP     & 80\%     & \multicolumn{1}{l|}{} & \multicolumn{1}{l|}{Proyecto 1} & \multicolumn{1}{l|}{Stand by}   & \multicolumn{1}{l|}{} & \multicolumn{1}{l|}{Proyecto 2} & \multicolumn{1}{l|}{Trabajando} \\ \cline{1-3} \cline{5-6} \cline{8-9} 
		Proyecto 2     & PG     & 25\%     & \multicolumn{1}{l|}{} & \multicolumn{1}{l|}{Proyecto 3} & \multicolumn{1}{l|}{Trabajando} & \multicolumn{1}{l|}{} & \multicolumn{1}{l|}{Proyecto 4} & \multicolumn{1}{l|}{Acabado}    \\ \cline{1-3} \cline{5-6} \cline{8-9} 
		Proyecto 3     & RP     & 55\%     &                       &                                 &                                 &                       &                                 &                                 \\ \cline{1-3}
		Proyecto 4     & PG     & 100\%    &                       &                                 &                                 &                       &                                 &                                 \\ \cline{1-3}
		Proyecto 5     &        & 0\%      &                       &                                 &                                 &                       &                                 &                                 \\ \cline{1-3}
	\end{tabular}
\end{table}

En sus inicios el desarrollo de software se realizaba a través de la empresa Elara, perteneciente a CyC. Con el paso del tiempo, con los proyectos en una fase más avanzada y ver la necesidad de más flexibilidad, se comenzó a desarrollar el software en el departamento de IT, apoyándose ocasionalmente en Elara.

La mayoría de proyectos son elaborados por un único miembro del equipo, por lo que no requieren una amplia planificación, Además, dado que el principal foco de trabajo es la asistencia al usuario y gestión de sistemas, no queda mucho tiempo para el desarrollo de nuevos proyectos más ambiciosos.

\paragraph{Mis sugerencias} 

Esta forma de gestionar proyectos no aporta una gran información, puesto que las cifras de progreso son subjetivas a sus desarrolladores y la información contenida en la pizarra no es actualizada a diario. Esto es fácilmente mejorable con un software especializado, como Microsoft Project.

En cuanto a la tendencia de realizar los proyectos en el propio departamento, lo considero una solución muy acertada, pues la flexibilidad y rapidez aporta un gran valor a la empresa, sobre todo el contar con ERP a medida, fácilmente moldeable a sus necesidades. Esta tendencia tiene un gran riesgo de gestión, que para solventarlo, se deben hacer grandes esfuerzos en realizar un código legible, estándar y fácil de mantener para quién no lo ha escrito, sobre todo si se esperan nuevas incorporaciones al departamento o la rotación de personal es alta.

También se podría mejorar en el departamento con la implantación de metodologías modernas de gestión, en mi opinión kanban encajaría muy bien con las características del departamento, haciendo además de gestor de proyectos. Esta metodología ayudaría a aportar mayor flexibilidad, promocionar el trabajo en equipo, estimular el rendimiento y aportar una mejor organización global del trabajo. \cite{miexperiencia}
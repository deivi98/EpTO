% !TeX spellcheck = es_ES
\subsection{Estructura}

\subsubsection{Organigrama}

\begin{figure}[h]
	\caption{Organigrama de plantas}
	\centering
	\label{orgGlobal}
	\includegraphics[width=0.65\textwidth]{organigramaGlobal.PNG}
\end{figure}

\begin{figure}[h]
	\caption{Organigrama planta}
	\centering
	\label{orgPlanta}
	\includegraphics[width=0.85\textwidth]{organigrama.PNG}
\end{figure}

\break

\subsubsection{Departamentos}

La fábrica cuenta con diferentes departamentos, a continuación, enumeraré de forma resumida sus labores en el día a día:

\begin{itemize}
	\item Producción: Planificar y controlar la fabricación, dar servicio al cliente, RRHH, Almacén P.C.P.
	\item Compras: Proveer componentes, logística de compra, negociación con proveedores y análisis de necesidades de compras.
	\item Ingeniería: Definición de métodos de trabajo, diseños útiles, lay-out, mejoras en la fábrica...
	\item Mantenimiento: Mentenimiento preventivo de medios, mantenimiento correctivo, reforma de máquinas, gestión de repuestos.
	\item Ingeniería de producto: Crear nuevas versiones, incorporar nuevas referencias, identificar componentes, diseño de piezas nuevas...
	\item Proyectos especiales, informática e I+D: Control de stocks, incorporación de nuevos proyectos, mentenimiento de sistemas informáticos, Hardware y sistemas...
	\item Calidad y reman product: Seguimiento de los medios de control, control de la calidad de documentación, asegurar calidad del producto...
	\item SSMA: Promover alto nivel de salud y seguridad de los empleados, prevención de riesgos laborales...
	\item Lean: Apoyo en la implementación de herramientas de lean manufacturing.
	\item RRHH: Recibimiento de personal, contratos, nominas, formación...
	\item Almacén y ventas: Preparación de producto, albaranes, facturación, distribución del producto acabado.
	\item Financiero: Contabilización de facturas, realización de pagos, cierres de mes, análisis de costes.
	\item Equipo comercial: Vender el producto.
	\item Encargados de taller: Coordinar la fabricación a pie de fábrica.
	\item Operarios de la fábrica: Elaboración del producto.
\end{itemize}

\subsubsection{Opinión}

En cuanto al organigrama de la Figura \ref{orgGlobal} vemos que es el típico de una Sociedad Anónima, en el que los accionistas están por encima de la empresa y son ellos quienes marcan el rumbo. También vemos una división en las diferentes divisiones del grupo, esta se origina en las políticas de diversificación optadas por el grupo a mediados del siglo 20.

En cuanto a la estructura organizativa de GKN Carcastillo, vista en la figura \ref{orgPlanta}, vemos la planta dividida en dos secciones NMA\footnote{Niche, Customer, Motorsport \& Aftermarket} y servicios a producción, aquí se plasman los diferentes departamentos junto a sus directores, de estos dependen directamente sus integrantes. En el organigrama no esta plasmada la zona de producción, asumiendo que este es solo de la zona de oficinas.

En mi opinión la estructura esta muy bien planteada, no hay un exceso de cargos directivos como ocurre en otras empresas y al no ser muy grande el dialogo continuo entre departamentos funciona muy bien, fomentándose así el buen ambiente laboral. \cite{miexperiencia, planacogida}


% !TeX spellcheck = es_ES
\documentclass{article}

%methadata
\title{
	{Memoria de proyecto}\\
	{\large Beca de Colaboración} \\
	{} \\
	{\includegraphics[width=8cm,height=6.5cm]{upna.png}}
}
\author{David González Portillo}
\date{20 de Mayo del 2020}

%packages
\usepackage[left=2cm, right=2cm, top=3cm, bottom=3cm]{geometry}
\usepackage[utf8]{inputenc}
\usepackage[pdftex]{graphicx} % Incluir fotos
\usepackage{url} % Incluir URLs
\usepackage{lmodern, textcomp} % Para poder poner el simbolo del Euro
\usepackage[spanish]{babel} % Etiquetas en Español
\usepackage{wrapfig} % Wrap images
\usepackage{enumitem} % Compact list

\graphicspath{ {images/} }

\newcommand{\foo}{\hspace{-2.3pt}$\bullet$ \hspace{5pt}}

\usepackage{fancyhdr}
\lhead{Titulación: Grado en Ingeniería Informática \\
		Beca de Colaboración \\
		Universidad Pública de Navarra}
\rhead{
	Curso académico: 19-20 \\
	\leftmark \\
	\rightmark
}
\setlength{\headheight}{34.5pt}
\pagestyle{fancy}

\begin{document}
	% Portada
	\begin{titlepage}
	\centering
	\includegraphics[width=0.15\textwidth]{upna.png}\par\vspace{1cm}
	{\scshape\LARGE Universidad Pública de Navarra \par}
	\vspace{1cm}
	{\scshape\Large Memoria proyecto\par}
	\vspace{1.5cm}
	{\huge\bfseries Beca de Colaboración\par}
	\vspace{2cm}
	{\Large\itshape David González Portillo\par}
	\vfill
	Tutor:\par
	Jose Ramón González de \textsc{Menvidil}

	\vfill

	{\large \today\par}
\end{titlepage} \clearpage
	
	% Índice de contenido
	\thispagestyle{empty}
	\tableofcontents \clearpage
	
	% Poner a 0 el contador de paginas
	\addtocounter{page}{-1}
	
	% Diseño de parrafos del cuerpo del documento
	\setlength{\parindent}{2.5em}
	\setlength{\parskip}{1em}
	
	% Introduccion
	\section{Introducción}
	% !TeX spellcheck = es_ES
\subsection{Qué es EpTO}

Epidemic Total Order Algorithm for Large-Scale Distributed Systems (EpTO) es un protocolo de orden total
o protocolo de consenso diseñado para sistemas distribuidos altamente escalables. Se trata de un algoritmo
epidémico que nos garantiza el orden total, la validez e integridad y, con una gran probabilidad, la entrega
de los mensajes. La idea del protocolo es un simple broadcast: los nodos se envían mensajes entre sí,
cada nodo a todos los demás. Mensajes que reciben de la capa superior (la aplicación). El objetivo es
que todos los nodos reciban los mismos mensajes exáctamente en el mismo orden.

\subsection{¿Para qué sirve?}

\begin{wrapfigure}{r}{0.50\textwidth}
	\centering
	\includegraphics[width=0.50\textwidth]{epTo-arquitectura.png}
	\caption{Arquitectura EpTO}
\end{wrapfigure}

Sus utilidadades son muchas. Dado que es un protocolo, puede ser implementado por debajo de muchas
aplicaciones que funcionen de forma distribuida y que tengan una gran cantidad de nodos y clientes,
aplicaciones que necesiten estar siempre disponibles para usarse y que toleren gran cantidad de
fallos. En definitiva, EpTO se reduce a la comunicación de los nodos de una red distribuida, de forma
que la aplicación o utilidad que se le de por encima, es abstraída de su funcionamiento.

\subsection{Porqué en NodeJS}

Si hay algo por lo que se caracterizan los sistemas distribuidos, es que son asíncronos. Están preparados
para funcionar a pesar de los retrasos, caídas y demás fallos. Se piensan para ser muy robustos a pesar de
toda complicación. NodeJS (Javascript) es un lenguaje asíncrono, pensado para entornos reactivos,
ocurrencia de eventos. Cuando se dispara un evento este se añade a la cola, donde se desapila el primero
y se ejecuta el código reactivo correspondiente, el cual puede desencadenar más eventos, y así
sucesivamente. Los programas terminan en cuanto no esperan ningún evento, cosa que puede no suceder
nunca.

\begin{figure}[h]
	\centering
	\includegraphics[width=0.4\textwidth]{nodejs-eventloop.png}
	\caption{Ciclo NodeJS}
\end{figure} \clearpage
	
	% Objetivos
	\section{Objetivos}
	% !TeX spellcheck = es_ES
\subsection{Generales}

Objetivos generales

\subsection{Personales}

Objetivos personales de este proyecto \clearpage

	% El proyecto
	\section{EpTO: El proyecto}
		\subsection{Primera fase: Aprendizaje}
		\subsubsection{NodeJS y ZMQ}

Durante el primer mes me dediqué a recordar NodeJS y ZMQ. Utilicé ambas durante la beca
que realicé el año pasado, la cuál también era en sistemas distribuidos. Para recordar
NodeJS utilicé el mismo libro que el año pasado \cite{therightway}.
El libro tiene una serie de capitulos con ejercicios básicos que explican progresivamente
todo lo necesario. ZeroMQ (ØMQ) \cite{zmq} es una libreria universal de código abierto de
comunicación mediante mensajes disponible en múltiples mensajes, entre ellos, Node. Es
la capa intermedia entre nuestro protocolo y el nivel de transporte. ZMQ se encarga de
todo lo necesario a nivel socket: envío y recepción de mensajes, creación y escucha de
sockets, serialización, conexión, restablecimiento de la misma... Hace que resulte muy
cómoda la implementación de la parte de comunicación entre nodos. Es una librería muy
reconocida y utilizada, además de que posee una amplia documentación.

\begin{figure}[h]
	\centering
	\includegraphics[width=0.7\textwidth]{zmq.jpg}
	\caption{Node.js y ZMQ}
\end{figure}

\subsubsection{Comprensión EpTO}

\begin{wrapfigure}{r}{0.40\textwidth}
	\centering
	\includegraphics[width=0.40\textwidth]{properties.png}
	\caption{Propiedades EpTO}
\end{wrapfigure}


Llegamos a la clave del proyecto, el algoritmo. Después de recordar Node y ZMQ, dediqué las proximas
semanas a estudiar el algoritmo, sirviéndome del artículo en el que había sido propuesto \cite{epto}.
En éste, se explica breve y claramente cada uno de los componentes del protocolo, cómo se demuestra
su correcto funcionamiento, y cómo se garanizan las cuatro propiedades propuestas: integridad, validez,
orden total y entrega probabilística. 

\paragraph{La idea}
La idea detrás de EpTO para la entrega probabilística es el problema de los hoyos y las canicas.
Los nodos son pensados como un conjunto de hoyos y cada mensaje como un conjunto de canicas. Los hoyos
a los que no caen canicas representan los nodos a los que sel mensaje no ha llegado debido a un
fallo. El objetivo es enviar el número de canicas por mensaje necesario para garantizar que la probabilidad de
que todos los hoyos hayan recibido el mensaje por lo menos una vez sea muy alta.

El protocolo consta de dos componentes principales: el componente de difusión y el componente de ordenación.


\paragraph{Componente de difusión}
\begin{wrapfigure}{r}{0.50\textwidth}
	\centering
	\includegraphics[width=0.50\textwidth]{disseminationcomponent.png}
	\caption{Componente de difusión}
\end{wrapfigure}
Este componente se encarga de la comunicación y difusión de mensajes, los cuales se denominan eventos.
Cada evento es un mensaje que se desea enviar a todos los nodos. El componente empaqueta múltiples eventos
en lo que llama BALL (lo que antes hemos entendido como canica), y la envía al resto de nodos. Antes de
comprender el funcionamiento completo necesitamos entender las variables utilizadas. Tenemos el conjunto de
todos los procesos correctos (view), K es el tamaño de la muestra de procesos a los que va a ser enviada
la próxima BALL, TTL (Time To Live) es el número máximo de saltos que un evento puede realizar (mismo
funcionamiento que los paquetes TCP-IP), delta es la duración de cada ronda, y nextBall es la proxima BALL
o conjunto de eventos próxima a enviar.

El funcionamiento es simple: cuando el proceso recibe un evento
nuevo desde la aplicación, lo añade a la próxima BALL. Cuando se recibe una BALL de otro proceso,
se analizan todos sus eventos de forma que si su TTL ya es muy alto se ignoran, y si no se actualizan
o/e introducen en la próxima BALL para ser reenviados. Cuando comienza la próxima ronda, se aumenta en
1 el TTL de todos los eventos pertenecientes a la próxima BALL, se selecciona un subconjunto de tamaño
K de todos los procesos correctos, y se les envía la BALL. Una vez hecho, se pasa la BALL al componente
de ordenación y se reinicia la BALL a conjunto vacío para la próxima ronda.

\clearpage

\paragraph{Componente de ordenación}
\begin{wrapfigure}{r}{0.60\textwidth}
	\centering
	\includegraphics[width=0.60\textwidth]{orderingcomponent.png}
	\caption{Componente de ordenación}
\end{wrapfigure}

Una vez que los eventos llegan al componente de ordenación se preparan y ordenan para entregarlos
cuando sea posible a la aplicación en orden total. Tenemos tres sencillas variables: recieved es el mapa de
eventos recibidos pero no entregados, delivered es el conjunto de eventos ya entregados y lastDeliveredTs
es el tiempo en el que se creo el último evento entregado. Dicho esto, el funcionamiento es el siguiente:
Para empezar, todos los eventos recibidos aumentan el TTL en 1, dado que acaba de terminar la ronda. Después,
para cada evento de la BALL, si no ha sido entregado ya y es posterior al último evento entregado,
se actualiza y/e introduce en el mapa de recibidos.

Después se añade a una lista los eventos que son
entregables por el momento (cuyo TTL es superior al límite), y se calcula el mínimo tiempo de creación
de los eventos que NO lo son. A estos primeros entregables restaremos aquellos cuyo tiempo de creación sea
posterior al mínimo de los que no son entregables, puesto que si un evento anterior a ellos no es entregable,
éstos tampoco lo son. Finalmente, ya tenemos los eventos que son realmente entregables a la aplicación,
así que se ordenan por proceso de origen y tiempo de creación y se entregan.

\paragraph{Reloj y sincronización}
La ordenación de los eventos o mensajes de éste algoritmo depende en gran medida del tiempo en el que
se generan. Para ello el artículo asume que utilizando un reloj global el orden se garantiza siempre,
aunque también dice que también puede funcionar con un reloj lógico perfectamente, y explica cómo.

\clearpage

		\subsection{Segunda fase: Implementación}
		\subsubsection{Diseño}

El diseño está planteado por módulos, cada uno con una función específica. La implementación ha resultado
mucho más sencilla y ha permitido que el control de versiones y la evolución del proyecto hayan sido
mucho más limpios y eficientes. Distinguimos los siguientes módulos con una breve descripción:

\paragraph{Client (Cliente)}
Implementa las funcionalidades básicas de un cliente. Conectarse con otros clientes
y enviar y recibir mensajes. Es el punto intermedio entre el protocolo y la capa de la aplicación.
Es decir, la aplicación tendrá uso de un cliente en cada máquina en la que se instale.

\paragraph{Process (Proceso)}
Implementa las funcionalidades más próximas al nivel de transporte. Conexiones con otros
procesos, utilización de ZMQ, recepción de mensajes serializados... También posee un
identificador único y está directamente ligado al cliente. En otras palabras, un cliente
siempre tiene un proceso, y un proceso siempre pertenece sólo y exclusivamente a un cliente.

\paragraph{Message (Mensaje)}
Objeto que almacena la información a nivel de la aplicación que contiene un mensaje. Implementa también
las funciones básicas necesarias para serializar y desserializar los datos de forma que sean transferibles
a través de la red.

\paragraph{Event (Evento)}
Objeto que almacena un mensaje y toda la información relevante del evento, como el tiempo de creación,
el proceso o cliente que lo ha creado, su TTL y su identificador único.

\paragraph{Ball (Canica)}
Objeto que representa un conjunto de eventos listos para ser enviados. También es totalmente serializable.

\paragraph{Clock (Reloj)}
Simple módulo que simula el funcionamiento del reloj y nos proporciona el tiempo. Puede estar implementado
de forma que sea un reloj global o uno lógico.

\paragraph{PSS (Peer Sample Service)}
Módulo que dado un conjunto de conexiones a procesos correctos y un tamaño de muestra, nos devuelve
un subconjunto aleatorio de las mismas.

\paragraph{Dissemination Component (Componente de difusión)}
Implementa todas las funcionalidades de dicho componente. Recepción y envío de canicas,
rondas..

\paragraph{Ordering Component (Componente de ordenación)}
Implementa todas las funcionalidades del componente de ordenación. Tan solo consta de dos
funciones, ordenar y entregar los eventos posibles, y comprobar si un evento es entregable.

\subsubsection{Arquitectura}

\begin{figure}[ht]
	\centering
	\includegraphics[width=0.7\textwidth]{graph.png}
    \caption{Arquitectura red de nodos}
    \label{graph}
\end{figure}

La mayoría de los protocolos de orden total tienen grandes limitaciones, y la primera de ellas suele ser
la escabilidad. EpTO trata de resolver este problema, ofreciendo una solución funcional para redes
distribuidas con un gran número de nodos. El problema cuando existen tantos nodos es que llega un punto
en el que es computacionalmente contraproducente enviar cada mensaje al resto de nodos, dado que la
cantidad de estos suele ser abrumadora. Y si no se envían a todos es difícil garantizar que el mensaje
llega finalmente a todos los nodos. EpTO intenta resolver éste problema mediante el reenvío contínuo de los
mensajes con un máximo determinado número de saltos siempre a un subconjunto (mucho más reducido,
y no computacionalmente contraproducente) aleatorio de procesos. Los procesos mismos se encargan de
propagar, no mejor dicho, el mensaje a toda la red.

En la figura \ref{graph} podemos ver un pequeño ejemplo de esto. Aunque todos los nodos saben de la
existencia del resto, cada vez que éstos reciban un mensaje que deban reenviar, lo harán sólo
a un subconjunto mínimo de ellos, hasta que finalmente y con las rondas necesarias, el mensaje
habrá sido recibido por todos los nodos (justo como explica el problema de los hoyos y las canicas).

\subsubsection{Recursos utilizados}

Para el desarrollo del proyecto he utilizado diversos recursos. Para el control versiones
he utilizado git, y para la implementación Visual Studio Code. Ambos entornos relativamente
nuevos para mi. También he utilizado npm (node package manager) para el despliegue del
proyecto y he aprendido LaTeX para redactar esta misma memoria. Además el proyecto entero está programado
en Typescript, a diferencia del año pasado, que desarrollé en Javascript puro. Typescript ofrece
la oportunidad de desarrollar de forma más clara, dado que es un lenguaje tipado, así que
el código será más fácilmente interpretado.

		\subsection{Tercera fase: Testeo}
		\subsubsection{Clustering}

Esto habla de como se ha testeado en clusteres

\subsubsection{Resultados}

Aqui se muestran los resultados y graficas del proyecto y su rendimiento

	\clearpage

	% Conclusión / valoración final
	\section{Conclusión}
	% !TeX spellcheck = es_ES

Una vez más, al igual que el año pasado, considero esta experiencia como una oportunidad
de inversión en mí mismo, además de posibilidad de ayudar en proyectos a la universidad,
en la cual he aprendido muchísimo. Esto me acerca y prepara cada vez más para la salida,
ya muy cercana, de la universidad. Además, estos dos últimos años, me ha permitido aprender
y desarrollarme en un círculo que no se enseña en la carrera, que es muy interesante y
me puede ser de gran ayuda fuera, y se trata de sistemas distribuidos.

He de decir que he tenido la suerte de permanecer los dos años en el mismo departamento,
con el mismo profesor y mismos compañeros, lo que ha hecho muy cómodo a la larga mi
estancia. Gracias a esto, todo lo relacionado a este año ha ido más rápido y he tenido
menos problemas, de forma que he podido abarcar un proyecto más grande y complejo. \clearpage
	
	% Referencias
	\thispagestyle{plain}

\begin{thebibliography} {0}
	\addcontentsline{toc}{section}{Referencias}
	\bibitem{epto} Miguel Matos, Hugues Mercier, Pascal Felber, Rui Oliveira, José Pereira. EpTO: An Epidemic Total Order Algorithm for Large-Scale Distributed Systems. \emph{2014}. \\ \url{https://haslab.uminho.pt/mmatos/files/p100-matos.pdf}
	\bibitem{matosthesis} Miguel Ângelo Marques de Matos. Epidemic Algorithms for Large Scale Data Dissemination. Programa de Doutoramento em Informática das Universidades do Minho, de Aveiro e do Porto. \emph{Julio de 2013}. \\ \url{https://haslab.uminho.pt/mmatos/files/miguel_angelo_marques_de_matos.pdf}
	\bibitem{therightway} Jim. R Wilson. Node.js the Right Way. Practical, Server-Side Javascript That Scales.
	\bibitem{zmq} ZMQ. An open-source universal messaging library. \\ \url{https://zeromq.org/}
\end{thebibliography}
\end{document}
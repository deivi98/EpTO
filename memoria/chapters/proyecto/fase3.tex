\subsubsection{Clustering}

En el laboratorio de la universidad disponía de un sistema de clústeres Proxmox con siete
máquinas distintas, en las cuales creé una serie de contenedores. Un contenedor es muy
similiar a una máquina virtual, realiza todas las funciones de una máquina independiente real,
o más bien, simula, de forma que es ideal para hacer pruebas de sistemas que luego realmente
sí van a estar en máquinas totalmente independientes. Los contenedores están completamente aislados
entre sí, permitiendome así testar el algoritmo como si de una red real se tratase.

Así pues, realicé múltiples tests y pruebas, desde redes con tamaño pequeño (7 nodos) hasta redes
de gran tamaño (700 nodos). Por supuesto, inicialmente tuve que lidiar con problemas que iban surgiendo,
sobretodo según aumentaba el tamaño de la red y la frecuencia con que se enviaban los mensajes entre
sus nodos. Progresivamente fui corrigiendo los errores en el código y resolviendo los problemas
generados en las pruebas, hasta que obtuve una versión perfectamente funcional.

\subsubsection{Resultados}

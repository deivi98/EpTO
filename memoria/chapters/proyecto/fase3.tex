\subsubsection{Clustering}

En el laboratorio de la universidad disponía de un sistema de clústeres Proxmox con siete
máquinas distintas, en las cuales creé una serie de contenedores. Un contenedor es muy
similiar a una máquina virtual, realiza todas las funciones de una máquina independiente real,
o más bien, simula, de forma que es ideal para hacer pruebas de sistemas que luego realmente
sí van a estar en máquinas totalmente independientes. Los contenedores están completamente aislados
entre sí, permitiendome así testar el algoritmo como si de una red real se tratase.

Así pues, realicé múltiples tests y pruebas, desde redes con tamaño pequeño (7 nodos) hasta redes
de gran tamaño (70 nodos). Por supuesto, inicialmente tuve que lidiar con problemas que iban surgiendo,
sobretodo según aumentaba el tamaño de la red y la frecuencia con que se enviaban los mensajes entre
sus nodos. Progresivamente fui corrigiendo los errores en el código y resolviendo los problemas
generados en las pruebas, hasta que obtuve una versión perfectamente funcional.

\subsubsection{Resultados}

Como se ha descrito inicialmente, el objetivo del algoritmo es conseguir que todos los nodos
de la red reciban los mismos mensajes exáctamente en el mismo orden. Para comprobar esto,
implementé una serie de tests que imprimían en ficheros por cada cliente los mensajes
según estos los recibían, y posteriormente comparaba los registros de los diferentes clientes.
En todas las pruebas realizadas, todos los registros coinciden al 100\%.

Hay que tener en cuenta, que dado que las pruebas se realizaron en un entorno más bien cerrado, un clúster con todas las
máquinas en la misma red, con una buena comunicación entre sus máquinas, es muy improbable la
pérdida de mensajes. En un sistema más grande, con más asincronía, sería de esperar la falta
de algunos mensajes (cosa totalmente normal). Los resultados para este proyecto son óptimos,
al menos desde punto de vista.
